Design and implementation defects that lead to inefficient computation widely
exist in software. These defects are difficult to avoid and discover.
They lead to severe performance 
degradation and energy waste during production runs, and are becoming 
increasingly critical with the meager increase of 
single-core hardware performance and the increasing concerns about energy
constraints. Effective tools that diagnose performance 
problems and point out the inefficiency root cause 
are sorely needed.

The state of the art of performance diagnosis is preliminary. Profiling
can identify the functions that consume the most computation resources, but can neither identify the ones that waste the most resources nor explain why.
Performance-bug detectors can identify
specific type of inefficient computation, but are not suited for
diagnosing general performance problems. Effective failure
diagnosis techniques, such as statistical debugging, have been proposed for
functional bugs. However, whether they work for
performance problems is still an open question.

In this paper, we first conduct an empirical study to understand how performance
problems are observed and reported by real-world users. Our study shows that
statistical debugging is a natural fit for diagnosing 
performance problems, which
are often observed through comparison-based approaches and
reported together with both good and bad inputs. We then thoroughly investigate
different design points in statistical debugging, including three different
predicates and two different types of statistical models, to understand which
design point works the best for performance diagnosis. Finally, we study how 
some unique nature of performance bugs allows sampling techniques to  
lower the overhead of run-time performance diagnosis without extending the
diagnosis latency.
