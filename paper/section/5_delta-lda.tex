



We also tried statistical debugging using the $\Delta$LDA model
together with the branch, return, and scalar-pair predicates. 
For branch predicates, we focus on predicates collected
at loop-condition branches here and we will refer to them as 
``Branch$_{\text{loop}}$'' in 
Table \ref{tab:in-house}.

As shown in Table~\ref{tab:in-house},
$\Delta$LDA model well complements the statistical debugging 
designs discussed earlier
(i.e., basic statistical model).
In 11 out of 12 cases where the basic statistical model fails to identify
good failure predictors, useful failure predictors are identified by
the $\Delta$LDA model.

Among the three different types of predicates, branch predicates are the
most useful --- help diagnosing 11 cases under $\Delta$LDA model. 
In general, when a loop-branch predicate $b$ is considered as a failure
predictor by the $\Delta$LDA statistical model, it indicates that $b$'s
corresponding loop executes many more iterations
during failure runs than during success runs.

%TODO Example



In eight cases, the loop ranked number one is exactly the root cause of 
computation
inefficiency. Developers fix this problem by (1) completely removing the
inefficient loop from the program (indicated by ``Remove the loop'' in 
Table \ref{tab:in-house});  
(2) reduce the workload of the loop (indicated by ``Reduce \# loop iterations'' in Table \ref{tab:in-house}); or
(3) remove redundancy
across loop iterations or across loop instances
(indicated by ``Combine loop iterations'' or ``Combine loop instances'' in Table \ref{tab:in-house}).

In three cases, the root-cause loop is ranked within top four (second, third, 
and fourth, respectively), but not number one. The reason is that the loop
ranked number one is actually part of the \textit{effect} of the performance
problem. For example, in GCC\#8805 and GCC\#46401, the root-cause loop
produces more than necessary amount of work for later loops to handle, which
causes later loops to execute many more iterations during failure runs
than success runs.

In one case, GCC\#12322, the root-cause loop is not ranked within top five
by $\Delta$LDA model. Similar with GCC\#8805 and GCC\#46401, the root cause
loop produces many unnecessary tasks. In GCC\#12322, these tasks 
happen to be processed by many follow-up
nested loops. The inner loops of those nested loops are all ranked higher
than the root-cause loop, %as they experience more significant iteration-number 
as they experience many more iteration-number 
increases from success runs to failure runs. 



Return predicates and scalar-pair predicates can also help diagnose some
performance problems under the $\Delta$LDA model, but their diagnosis
capability is subsumed by branch$_{\text{loop}}$ predicates in our evaluation,
as shown in Table \ref{tab:in-house}.
For the six cases when a scalar-pair predicate $p$ is identified as a good 
failure 
predictor, $p$ is exactly part of the condition evaluated by a corresponding
branch$_{\text{loop}}$ failure predictor.
For the seven cases when a function-return predicate
$f$ is identified as a good failure predictor, $f$ is ranked high by the
statistical model because it is inside a loop that corresponds to a highly
ranked branch$_{\text{loop}}$ failure predictor.

\begin{table*}[tb!]
\begin{adjustwidth}{-.5in}{-.5in}
\small
\centering
{
\begin{tabular}{|l|c|c|c|c|c|c|}
\hline
\multicolumn{1}{|c|}{Fix Categories} &Apache&Chrome&GCC&Mozilla&MySQL&Total\\
\hline
Total \# of loop-related bugs    & 10 & 4 & 7 & 12 & 10 & 43 \\
\hline

Remove the loop                                                       &0&1&2&4&3&10    \\
Combine loop instances (removing cross-loop redundancies)             &3&2&0&4&1&10   \\
Reduce \# loop iterations (reduce the workload of the loop)           &0&0&4&2&2&8    \\
Combine loop iterations (removing cross-iteration redundancies)       &6&1&1&1&1&10   \\
Others                                                                &1&0&0&1&3&5    \\
\hline
\end{tabular}
}
\end{adjustwidth}
\caption{Fix strategies for loop-related bugs}
\label{tab:loop-rootcause}
\end{table*}



\paragraph{Comparing with the profiler} 
$\Delta$LDA model is good at identifying root causes located inside loops. 
Since functions that contain loops tend to rank high by profilers, profilers
perform better for this set of performance problems than the ones discussed
in Section \ref{sec:inhousebasic}. In comparison, statistical debugging still
behaves better.

In terms of identifying root causes, $\Delta$LDA model always ranks the
root cause loop/function equally good (in 7 cases) 
or better (in 4 cases) than profilers. There are mainly two reasons that 
$\Delta$LDA is better. First, sometimes, the root-cause loop does not take
much time. They simply produce unnecessary tasks for later loops to process.
For example, in GCC\#8805, the function that contains the root-cause loop
only ranks 20th by profiler. However, it is still ranked high by $\Delta$LDA
model, because the loop-iteration-number change is huge between
success and failure runs. Second, sometimes, functions
called inside an inefficient loop take a lot of time. 
Profilers rank those functions high, while those functions actually do not
have any inefficiency problems.

Considering call-stack functions in the profiling results (``Stack'' column
in Table \ref{tab:in-house}) does not make profiler much more useful.
For example, the root cause function of GCC\#46401 ranks fifth in
the profiling result. This function is also one of the callers' callers of the
top-ranked function in the profiling results. However, since the profiler
reports three different callers, each having 1--3 callers,
for the top-ranked function, the effective ranking
for the root-cause function does not change much with or without considering
call stacks.
